\newcommand{\R}{\textbf{[ref]}}
\newcommand{\Rlots}{\textbf{[lots of ref]}}


The term `jets' are used to refer to collimated ejecta having opening angle less than $15^{0};$ they are at least four times as long as they are wide \citep{Bridle_&_Perley-1984-ARAA}. The flat-spectrum radio emission, i.e. the energy-independence of the observed flux with energy, is a distinguishing characteristic of relativistic jets from compact objects, for example in the case of Cygnus X-1 \citep{Pandey_et_al.-2007-AAp,Fender_et_al.-2000-MNRAS}.

Protostellar outflows are thought to be associated with the accretion disks around forming stars \R, studied mostly in the infrared wavebands. In black hole sources, jets make their presence felt across the electromagnetic spectrum at different stages of a black hole. Whereas long Gamma Ray Bursts (hereafter GRBs)  \R  are thought to be powered by the ejection events during the formation of stellar mass black holes \R, short Gamma Ray Bursts \R  are thought to be electromagnetic counterparts of gravitational waves, produced during mergers of compact objects (Neutron Stars and Black Holes) \Rlots. Quasars \R are extremely luminous point sources at cosmological distances, thought to powered by the accretion on to supermassive black holes at the centres of old (i.e. high redshift) galaxies \R, and form a subset of a more larger sample of black hole sources powered by accretion, collectively termed the Active Galactic Nuclei \Rlots or AGN. The AGN sample consist of extragalactic sources ranging from Seyfert galaxies \R, Blazars \R and BL Lacs \R. Unlike protostellar outflows, jets from black hole sources are known to be relativistic and particle accelerators with powers far beyond the present reach of artificial particle accelerators.

Pulsars are now known to be cosmic lighthouses, with their relativistic jets pointing towards the observer's line-of-set at extremely precise time-periods due to their precession about the rotation axis of neutron stars, powered either by accretion of matter onto a neutron star from a companion star (millisecond pulsars \R) or by the rotation power of the neutron star as it spins down in a high magnetic field environment \R. In fact, magnetic fields are thought to play a very crucial role in the emission associated with jets. The high brightness temperatures associated with jet-emission points towards the particle-spectrum being non-thermal, radiating only a fraction of their total energy as they cool down \R. Such non-thermal radiation are known to be anisotropic, i.e. boosted towards the observer \Rlots, and this plays a crucial role in the observation of such relativistic jets. In analogy to Quasars, a new sample of jets were discovered in the 1990s \Rlots, powered by accretion onto stellar-mass black holes. Each of these black holes are fed by a stellar mass companion star tightly bound in orbit around it, and this sample has since been termed `Microquasars' \R. More recently, in 2007 \textbf{[confirm]}, a new class of events called Fast Radio Bursts (FRBs) have been discovered \R. Since their energetics and variability are similar to that of GRBs, it is universally accepted that they are also associated with relativistic jets coming from compact objects. However, the exact mechanism of these exotic sources are highly debated \Rlots.

\section{The observational front}

\section{Jet models}
Generalized jet models, different kinds.

\section{Central engines}
Accretion phenomenon and jets are ubiquitously related \emph{elaborate the basic physics thought to be at play \Rlots.}