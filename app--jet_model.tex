Here we lead the interested reader to the deduction of Equation \ref{eq:C_3} from Equation \ref{eq:C_3_double_integral} in the low-energy regime as discussed in the text.

Firstly, we note that the integrand is even over $ \psi, $ hence it can be written as
\begin{equation}
C_3(p) = 2 \int_{0}^{1}d\psi\, T(\psi),
\label{eq:C_3_from_Tpsi}
\end{equation}
where
\begin{equation}
T(\psi) = \int_{0}^{\infty} d\xi \; \xi^{3/2} \left[ 1 - \exp \left\{ -\xi^{-\frac{p+4}{2}}\left(1-\psi^{2}\right)^{1/2} \right\} \right],
\end{equation}
which formally diverges. However, the divergences can be shown to cancel, by defining the upper-limit to be $q$ and taking the limit of $ q \rightarrow \infty , $ i.e.
\begin{equation}
T(\psi) = \lim_{q \to \infty} \int_{0}^{q }d\xi \; \xi^{3/2} \left[ 1 - \exp \left\{ -\xi^{-\frac{p+4}{2}}\left(1-\psi^{2}\right)^{1/2} \right\} \right],
\end{equation}
and carrying out the above integral by parts, with $ \xi^{3/2} $ being `taken out' of the integral, which yields
\begin{eqnarray}
T(\psi) & = & \frac{p+4}{5} (1-\psi^2)^{1/2} \int_{0}^{\infty}d\xi\,\xi^{-\frac{p+1}{2}} \exp \left\{ -\xi^{-\frac{p+4}{2}}\left(1-\psi^{2}\right)^{1/2} \right\} \\
 & = & \frac{2}{5} (1-\psi^2)^{ \frac{5}{2(p+4)} } \int_{0}^{\infty}du \, u^{ \frac{p-1}{p+4} } \, e^{-u},
\end{eqnarray}
the last step being obtained by the substitution $ u = \xi^{ - \frac{p+4}{2} } (1-\psi)^{1/2} , $ to give
\begin{equation}
T(\psi) = \frac{2}{5} (1-\psi^2)^{ \frac{5}{2(p+4)} } \Gamma\left( \frac{p-1}{p+4} \right),
\end{equation}
which can be finally plugged into Equation \ref{eq:C_3_from_Tpsi} to obtain the final result.