Driven by the negative prospects of observing GRBs with \A -WFI [see Section \ref{subsec:predictions_for_soft_Xray_instruments--long}], we ask the following question instead: What is the prospect of \A -WFI to observe Tidal Disruption Events [TDEs]?

We first note that there are many issues with TDE science:
\begin{enumerate}
\item It is difficult to identify TDEs. In the optical wavelengths, it is difficult to distinguish them from supernovae. Even amongst so-called confirmed TDE candidates, there is no uniformity in the slope of the lightcurve [which may vary for a given TDE], or their spectral properties. Subsequently, multiple claims have been made that a certain extragalactic source is a TDE candidate via purely spectral fits, but there is no confirmation as such for these sources. On the other hand, theoretical studies claim that a good fraction of TDEs [$\sim 10 \%$] of AGN activity is due to TDEs; that the TDE-AGN activities span a continuous spectrum than a discrete difference in observables.

\item The intrinsic number of TDEs are much smaller than other transients [like GRBs]: $10^{-5}$-$10^{-4}$ per year per galaxy \citep{Auchettl_et_al.-2018-ApJ}. This is significantly lesser than the progenitor mass available for GRBs, for example.
\end{enumerate}

As such, severe instrumental effects are present in the sample of the TDEs themselves. There are around $70$ TDE candidates \citep{Auchettl_et_al.-2017-ApJ}, out of which a significant fraction may not be TDEs. Confirmed TDEs counts is $\sim 20$ in all these years of observation. Here is where an instrument with the probability of detection of $1$ TDE per year may make a significant impact to the TDE science. Roughly, the TDE ``activity'' time is $\sim \frac{1}{3}$ year, so follow-up of the same TDE in different wavelengths, and hence being able to confirm the fact that it is a TDE, is important yet observationally not expensive for such an instrument. Currently, TDEs have been observed either at optical wavelengths, or at soft X-rays. Aside the current optical surveys, a good soft X-ray monitor has the capability of increasing the TDE sample. \emph{MAXI} is a wide-field soft X-ray monitor which has detected 4 TDEs in 37 months of data \citep{Kawamuro_et_al.-2016-PASJ}. Other important instruments have been \s -XRT and the XMM-Newton slew survey. The luminosity function [mathematical definition different than the GRB case, although in the same spirit] has been studied by both \cite{Kawamuro_et_al.-2016-PASJ} and in much more detail by \cite{Auchettl_et_al.-2018-ApJ}. Whereas the former is severely limited by the low statistics of the sample, the latter somewhat alleviates this problem. The selection effects at different luminosities, however, cannot be mitigated. Acknowledging these effects, they have still been able to make some important statements:

\begin{itemize}
\item At $\LX < 10^{44} \, \ergpersec$, the contribution of both jetted and non-jetted TDEs to the AGN LF is significant, specially at low redshifts [$z < 0.4$], in line with theoretical predictions \citep{Milosavljevic_et_al.-2006-ApJ}. However at higher redshifts the contribution becomes less significant.

\item The observed TDE LF has virtually no contribution from the non-jetted TDEs at $\LX > 10^{44} \, \ergpersec$, as also expected theoretically [from the BH mass limit].

\item The observed TDE LF flattens for both the jetted and non-jetted TDEs, thus deviating from the theoretical predictions, for $\LX \sim 10^{40}$-$10^{42} \, \ergpersec$, which is mostly likely an observational artifact. The TDEs supposed to populate this region can be explained by the ``veiled'' TDEs \citep{Auchettl_et_al.-2017-ApJ}.

\item The overall behaviour points to the fact that the observed TDE LFcould very well converge to the theoretical prediction, with future observations. This has an important consequence, as discussed below.
\end{itemize}

The intrinsic TDE source rate derived from the said theoretical study implies a significantly higher rate of stellar disruptions near central supermassive black holes, than those inferred from observations in the optical/UV. This raises another important question: Is the rate of TDEs inferred from optical/UV studies an order of magnitude smaller than those from X-ray observations?

If the above is true, then the prediction for the observable rate for any soft X-ray instrument can increase by a factor of $10$. This is supported by the recent finding of \cite{Tadhunter_et_al.-2017-NatAst}.

Taking all these into account, it can be argued that if a particular instrument can detect and follow-up $1$ TDE $\py$, it can significantly impact the field. Not only that, it can alleviate the instrumental selection effects at the $10^{40}$-$10^{42} \, \ergpersec$ plateau, if sensitive at an energy range that may pick up TDEs with such luminosities. Ideally, one would like a wide-field soft X-ray monitor. A trade-off might be obtained by having an instrument with much better sensitivity but lower FoV. Such an instrument is \A -WFI. It has the potential to significantly impact this field if the number of galaxies it can survey is around $10^3$-$10^4$ times larger than the current instruments.

For focussing instruments on-board \X, \C, \A, the sensitivity scales with time as $t^{-1}$. As for TDEs, the time to be taken for integration is uncertain. But since WFI is an instrument that will not be staring at the same field for months, it is safe to assume that the sensitivity will be maximum two order higher than that for GRBs. Taking into effect the mass and luminosity function of galaxies \citep{Conselice_et_al.-2016-ApJ}, and given that the intrinsic TDE rate is much smaller \citep{Auchettl_et_al.-2018-ApJ}, the number of galaxies available for survey will not be enough for WFI to make any significant stride in this field.

There is another criticism against WFI as a TDE detector: Even though it is going to observe deeply when it its narrow FoV, without having an existing catalogue of the steady sources in that field from a previous survey, it will be impossible to understand whether a given source is a steady source or a transient. This problem does not arise for GRBs because the prompt emission of GRBs are really short [maximum around $100$ s] and extremely significant [at least for LGRBs], but a large number of TDEs are still confused with supernovae as these two classes behave similarly, both temporally and spectroscopically. Only if e-ROSITA can survey deeper than WFI and complete the full-sky catalogue before WFI starts observing, will it be practically possible to pitch WFI as a TDE detector.

\begin{checkit}
The results from this literature survey and basic calculations were communicated to Professor David Burrows and Dr. Pragati Pradhan of the Pennsylvania State University, State College, Pennsylvania, the USA.
\end{checkit}