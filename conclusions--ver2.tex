When I started my Thesis work, \AS\ had already gone through its performance-verification phase. India's first satellite solely dedicated to astronomy, it had however seen a long delay before being flown. For someone who had not seen the multiple technical challenges that it had gone through, working with the data to make significant science cases itself seemed like a big challenge. I started with small steps as a part of the CZTI collaboration, taking up the task of comparing simulations via an existing code, with data. Having established that CZTI does have the capability of localising transient events incident at large angles to the pointing axis and well outside the primary field of view [FOV], an understanding emerging from this analysis caught my interest. It was seen that the noise in the data was not very well understood, and one or more components of events generated by non-astrophysical sources was affecting the analysis of CZTI data. A question arose: given the current sensitivity of the instrument, the FOV to off-axis sources, and the wavelength-range, how many GRBs could \AS -CZTI detect per year? Would it be as high as that of \f -GBM, a truly broadband, wide FOV instrument? Or would it be much smaller, close to that of \s -BAT, a detector that reports GRBs seen only within its narrow primary FOV, in a much smaller waveband? Given that the specifications of \AS -CZTI lay somewhere in between these two extremes both in FOV and wavelength coverage, surely the answer lay somewhere in between as well. But what exactly is the expected rate of detection? And how much does the uncharacterised noise affect this number? Could the understanding of the noise be improved so as to effectively improve the sensitivity of the instrument, and hence also this number? These questions demanded answers. Below is summarised some of the understandings that have emerged as a result of my Thesis work.

\begin{description}

\item[Observed long GRBs are a homogeneous population.]  \hfill \\
The existing catalogue of long GRBs forms a homogeneous sample of bursts that can be used to model the intrinsic distribution of the population as well as their cosmological rates. However, due to various reasons, the full potential of this large, existing sample had not been used in the literature.

\item[Long GRBs cannot be used as standard candles.] \hfill \\
The fact that GRBs are extremely luminous, means that they are seen across extremely large distances across the universe. Correlations of observed GRB properties had been used in the literature to `standardise' long GRBs as distance-indicators. As a part of the study of the long GRB population, it was shown that although the strongest reported correlation was indeed quite strong, there was significant scatter, even after taking care of observational errors. This meant that the distances to individual LGRBs could not be predicted accurately, thus ruling out LGRBs as distance-indicators.

\item[That does not rule out the statistical predictions for the entire sample.] \hfill \\
The correlation could however predict statistical properties with accuracy consistent with observed errors. This meant that the distance-indication, although not valid for individual GRBs, could be considered seriously in the statistical sense, thus alleviating the problem that most GRBs does not have measured redshifts. The luminosity of GRBs could be constructed for the entire sample, allowing the modelling of the luminosity function.

\item[The population models predict observable rates of any new detector.] \hfill \\
As the existing catalogue of \s\ and \f\ GRBs were used to construct the true all-sky cosmological variation of the GRB formation rate, it now facilitated answering the reverse-question: Given this true rate, how would a given GRB monitor sample it? Reliable answers could be arrived at for \AS -CZTI. Topical predictions were made for future instruments.

\item[The short GRB sample is also homogeneous.] \hfill \\
Since the number of short GRBs observed are much smaller than long GRBs, could the same mechanism be applied to them as well, or are we statistically limited? The answer turned out to be a yes, given the novel modification of an existing method in the literature. One needed to take care of instrumental detector sensitivity of the \emph{CGRO}-\B\ detector, to increase the sample such that it was statistically significant.

\item[The redshift measurement of short GRBs have selection bias.] \hfill \\
Unlike the case of long bursts, however, the short burst sample with measured redshifts were shown to have selection bias. This ruled out in one stroke, all results in the literature about SGRB populations that used only the sub-sample of redshift measured SGRBs to model the luminosity function.

\item[The short GRB distribution can also be constrained.] \hfill \\
A simple modification of the LGRB method allowed stronger constraints on the SGRB population and their formation rate, compared to widely different claims in the literature.

\item[It predicts the observable rate of binary neutron star mergers.] \hfill \\
The scenario could now be turned around to ask: How many binary neutron star mergers can the existing and upcoming gravitational wave networks detect? The answer is: at least twice per year from the next runs, which has now commenced. Most likely, the number is much larger.

\item[Each binary neuron star merger may produce a short GRB.] \hfill \\
Despite the presence of tension between the binary neutron star merger rates inferred here with the rate inferred from a Bayesian analysis of the one and only case of neutrons stars merger observed till date, the errors on both of these allow the scenario that each such coalescence causes a short GRB. Only time will tell, upon the improvement of constraints on both these numbers, whether such a hypothesis is true or the tension becomes stronger.

\item[\AS-CZTI misses out on a number of GRBs that it should detect.] \hfill \\
As inferred from the localisation exercise of the first GRB detected by \AS -CZTI, noise affected CZTI data. On being conservative about detections and rule out false-detections, and in the absence of automated detection software because the noise is not fully understood, CZTI detection of both long and short GRBs was smaller than predicted by the sample studies of the two classes.

\item[The noise component causing this is high energy cosmic rays.] \hfill \\
Unreported earlier, the low-rate, high energy cosmic ray component in the data affected it in a manner different from the earlier observed higher-rate, lower energy cosmic ray component. This urgently required an automated detection of this component from all CZTI data.

\item[A simple yet robust algorithm eliminates this component.] \hfill \\
Thoroughly tested on a large sample of diverse data across months, this algorithm leaves GRB photons alone while eliminating the high energy cosmic ray component. Hence it is crucial to include this in the data analysis pipeline, especially to improve the short GRB sample detected by \AS -CZTI.

\item[The anisotropy pattern of relativistic jets is not very well-understood.] \hfill \\
It is a well-known fact that relativistic beaming affects the inferred rates from a sample of GRBs. However, the angular distribution of relativistic jets is not very well-studied. A well-known model of the generation of relativistic jets is studied in detail to make theoretical calculations of the angular distribution. It is applied on a galactic Microquasar, but the case of highly variable, short-lived GRB jets requires significant numerical complexity.

\item[A sample of GRBs have different durations in \f\ and \s.] \hfill \\
A careful investigation of the sample of GRBs common to both \f\ and \s\ reveals a tension in the observed duration of a very small sub-sample. The difference in sensitivity of the two instruments creates this artefact for this sub-sample, which corresponds to GRBs with lower energy secondary flare[s] that are detected by \s\ but missed by \f.

\end{description}

%During the years when this Thesis work was being carried out, the astronomy community has seen significant new discoveries, including the imaging of the vicinities of a black hole, reported very recently \textbf{[ref]}. The detection of one ultra-high energetic neutrino from a blazar was reported a few months back \textbf{[ref]}. After a long wait, the first confirmed case of gravitational waves from binary black hole merger was announced in 2015. Although no electromagnetic counterpart of this source was found, the first detection of the merger of a pair of neutron stars was coincident with a short Gamma Ray Burst, putting the long-held association of short GRBs with neutron star mergers on an empirical footing. Global collaboration across wavelengths led to the detection of electromagnetic emission from this source at wavelengths from the X-rays to the radio.

I have demonstrated that \AS -CZTI needs improvements on its data-analysis pipeline along suggested lines to be able to make significant contributions to the GRB field. A future Indian mission dedicated to GRB science, \D, has great prospects in this new era of multi-messenger astronomy. There are also questions in the GRB phenomenology that needs answers, for example what fundamental physics determines the GRB luminosity functions.

