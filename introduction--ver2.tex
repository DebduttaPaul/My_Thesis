\newcommand{\R}{\textbf{(ref)}}
\newcommand{\Rlots}{\textbf{(lots of ref)}}


Compact objects -- white dwarves [WDs], neutron stars [NSs], and black holes [BHs] -- are theorized to be end-points of stellar evolution, and around $10^{8}$ to $10^{9}$ stellar-mass [$ M \sim 1 $-$ 10 \, M_{\odot} $] compact objects are estimated to reside in the Milky Way \citep{Timmes_et_al.-1996-ApJ}. The first strong evidence of a compact object came from the study of an optical star \citep{Webster_&_Murdin-1972-Nature}, hypothesized to be a binary system consisting of a BH and an optical companion in tight orbit around each other \citep{Bolton-1972-Nature-Identification}. This system is known as Cygnus X-1, the brightest source of X-rays in the Cygnus constellation, discovered in 1965 \citep{Bowyer_et_al.-1965}. The presently observed population of X-ray binaries in our galaxy is $\sim 200$ \citep{vanParadijs-1995-Catalogue}.

About $10 \%$ of the observed X-ray binary population are `radio-loud' \citep{Hjellming_&_Han-1995-H&H_Catalogue}, i.e. they emit in the radio wavelengths, and about $10$ of these have shown evidence of apparent superluminal motion extending over $\sim 0.1$-$1$ pc length scales [for example, GRS 1915+105 \citep{Mirabel_&_Rodriguez-1994-Nature,Mirabel_&_Rodriguez-1999-ARAA}]. Apparent superluminal motion over $\sim$ kpc scales are also seen from external galaxies, associated with the central active supermassive BH \citep{Zensus-1997-ARAA}. The observed X-ray emission is generally assumed to come from an accretion disk created by the in-falling matter from the companion star and heated by viscous dissipation \citep{Shakura-1972,Shakura_&_Sunayev-1973-AAp,Pringle-1981-ARAA}, while the radio emission is ascribed to non-thermal synchrotron emission from accelerated electrons in relativistic jets emitted from the accretion disk, carrying away angular momentum \citep{Mirabel_&_Rodriguez-1998-Nature,Fender_et_al.-1999-ApJL,Dhawan_et_al.-2000-ApJ}.

The term `jets' refers to collimated ejecta having opening angle less than $15^{\circ}$: they are at least four times as long as they are wide \citep{Bridle_&_Perley-1984-ARAA}. Jets are ubiquitous, powering a large class of sources: Gamma Ray Bursts [GRBs], associated with the formation of stellar mass BHs; Active Galactic Nuclei [AGN] consisting a range of sources like Quasars, Blazars, etc. powered by supermassive black holes [SMBHs] at the centres of galaxies; Pulsars, which are jets from NSs beaming towards the observer at a specific period; Microquasars, jets from stellar mass BHs accreting matter from a companion star. The answer to the origin of Cosmic Rays is believed to be relativistic charged particles ejected in jets from compact objects.

Contrary to outflows from star-forming cores called Protostars, jets from compact objects like NSs and BHs are always relativistic. The gravitational energy extracted from the central object depends on its compactness, that is on $1/r$, where $r$ is the distance of the matter in the accretion disk around the object. Hence, the energy-scales in the neighbourhood of Protostars are studied via the physics of atoms and molecules, whereas the energy-scales in the neighbourhood of compact objects leads to dense plasmas and intense magnetic fields, making them all the more interesting. Relativistic jets from compact objects serve as particle accelerators with power far beyond the present reach of artificial particle accelerators.

The major problems related to the physics of relativistic jets can be classified into the following: [1] jet launching mechanism; [2] jet composition. There are various models that explain the extraction of energy from compact objects \citep{Blandford_&_Znajek-1977-MNRAS,Blandford_&_Konigl-1979-ApJ,Blandford_&_Payne-1982-MNRAS}, but in practice, the extreme complexity of the problems make them computationally expensive, and only recently are simulations catching up with observations, that too with many caveats \,\Rlots. On the other hand, the composition of the jet, their nature of being ballistic or continuously fed by a central source of energy \,\R\, -- called the `central engine' \,\Rlots, the effect of their eventual deceleration \,\R, are all matters of continuous debate both in the theoretical as well as observational front.

NSs are limited from above to a few solar masses \,\R, but BHs range widely in masses from a few solar masses to billions of solar masses. Indeed, the galactic X-ray binaries are termed `Microquasars' \citep{Mirabel_&_Rodriguez-1998-Nature} to suggest the universality of the underlying physics across an enormous scale of BH masses, ranging from the stellar-mass black holes in the binary systems to the supermassive black holes at the centres of active galaxies. Indeed, the length and time-scales of the accreting BH systems are proportional to the mass of the central BH \citep{Sams_et_al.-1996-Nature}, which make them ideal sources to study the accretion and jet physics. The wide range of masses however results in a wide range in the properties of the jets from BHs, particularly in the Lorentz factor [$\Gamma$], from a few in Microquasars \,\R\, to decades in Blazars \,\R, to hundreds and even thousands in GRBs \,\R. Moreover, the extraction of the spin energy from the proximity of the BHs leads to relatively larger $\Gamma$s of the resulting jets \,\R. Whereas in NSs the physics of the NS surface and the surrounding magnetic fields determine the jet physics, the jets from BHs probe extreme gravity very close to the central BH.

AGN affect the evolution of galaxies and galaxy clusters. Known as the `AGN feedback' \,\R, consensus has emerged over the last decade that it plays a significant role in the growth of black holes at the centres of galaxies at the very high redshift universe \,\Rlots. AGN can outshine their parent galaxies, being orders of magnitude more luminous, thus letting us probe the high-redshift universe. Recent studies claim to have extended the distance-ladder of cosmology using Quasars, a special class of AGN \,\R.

Another beacon of the high redshift universe are GRBs \,\Rlots. Whereas long GRBs [lGRBs] are powered by ultra-relativistic jets of matter emitted from the environment of high-mass stars collapsing directly into black holes \,\Rlots, short GRBs [sGRBs] are also powered from the environment of NSs that merge together to form either a NS or a BH, releasing the energy initially stored in the high magnetic fields of the NSs \,\Rlots. Thus, the unprocessed hard X-ray emission of GRBs from the immediate environment of a nascent BH \,\R\, offers an unique probe into the launching mechanism of jets. The study of the timing, spectroscopy, and polarization of the hard X-rays emitted from GRBs is thus extremely crucial in unravelling the mysteries of jet launching and jet composition.

More recently, in 2007 \,\R, a new class of events called Fast Radio Bursts [FRBs] have been discovered \,\R. Since their energetics and variability are similar to that of GRBs, it is universally accepted that they are also associated with relativistic jets coming from compact objects. However, the exact mechanism of these exotic sources are highly debated \,\Rlots.